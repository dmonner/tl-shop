\documentclass[12pt]{article}
\usepackage{amsmath, amssymb, amsthm, alltt, graphicx, subfigure}
\usepackage[top=1.5in, bottom=1.25in, left=1.25in, right=1.25in]{geometry}
\begin{document}
\begin{titlepage}

\begin{center}


\vfill
{\LARGE \textsc{TL-Shop}: Temporal Logic Control Rules in a Hierarchal Task Network Planner}\\[5cm]

{\Large Term Project Report for CMSC 722, Spring 2007}\\[3cm]


% Author and supervisor
\begin{minipage}{0.4\textwidth}
\begin{flushleft} \large
\emph{Author:}\\
Nathaniel Ayewah
ayewah@cs.umd.edu
\end{flushleft}
\end{minipage}
\begin{minipage}{0.4\textwidth}
\begin{flushright} \large
\emph{Author:} \\
Derek Monner
dmonner@cs.umd.edu
\end{flushright}
\end{minipage}

\vfill

% Bottom of the page
{\large \today}

\end{center}

\end{titlepage}


\section{Introduction}
Hierarchal Task Network (HTN) planning requires describing a goal state as a 
set of high-level tasks which will achieve that state, and involves achieving 
the required tasks by recursively breaking them into smaller and smaller 
subtasks \cite{nau2004apt}. Put simply, HTN planning, describes in detail a 
subset of  solutions the planner is allowed to try. On the other hand, the 
control rule planning paradigm is a standard forward search procedure enhanced 
by control rules written in a modal logic such as Linear Temporal Logic (LTL) 
that are used to prune the search space by describing which states are 
undesirable or will never lead to a solution \cite{nau2004apt}. In short, 
control rule planning tells the planner which branches not to follow.

The product of this work, a planner called TL-SHOP, combines these two 
paradigms in an effort to increase the flexibility afforded to writers of 
planning domains and problems.

\section{Related Work}
TLPlan\cite{bacchus1996utl} is a well-known implementation of the control rule 
planning paradigm. Our approach to processing Linear Temporal Logic is based on 
a simplification of the strategy used in TLPlan\cite{nau2004apt}.

JSHOP2\cite{ilghami:dj} is an existing state-of-the-art domain-configurable HTN 
planner written in Java. JSHOP2 is unique in that it compiles a planning domain 
and problem into a domain-specific planner in order to increase efficiency 
\cite{ilghami2003gas}. TL-SHOP builds upon JSHOP2 by adding temporal 
constraints in Linear Temporal Logic at several levels, as discussed in the 
following section.

\section{Implementation}
We augmented JSHOP2 with control rules in three simple ways. First, TL-SHOP 
allows the addition of global control rules to the specification of the 
planning domain. As time progresses, if any of the control rules become 
verifiably false, the planner explores no further down that path. This behavior 
is analogous to the way control rules are expressed in TLPlan, and allows us to 
prune partial plans whenever a new operator is inserted (as this changes the 
current state on which the planner is focusing).

TL-SHOP allows a second type of control rule which is a simple extension of the 
first: problem-specific control rules. These rules are more or less treated the 
same as global domain control rules, except that they can be changed from 
problem to problem without the user needing to change the domain specification. 
They are meant to allow users to impose additional, non-domain-specific 
constraints. For example, in the well-known blocks world domain, a user could 
specify that a certain block must never be placed on the table. This control 
rule is not appropriate for the blocks world domain itself, since it may remove 
some or all valid plans, but in effect it creates a customized subdomain for 
this particular problem instance.

Finally, TL-SHOP allows the specification of method and operator postconditions 
as LTL formulas. When an operator or method with a postcondition is applied, 
its postcondition is added to the set of applicable control rules for every 
path in the search space below the applied method or operator. Returning to the 
blocks world domain for another example, the $stack(a, b)$ operator might have 
the postcondition $\square on(a, b)$ (``always $on(a, b)$''), which is a simple 
but effective way to satisfy the constraint that if one stacks a block, it 
should be in its goal location.

The following sections will cover the main pieces of the implementation in 
detail, as well as some limitations of the software.

\subsection{Extended Domain and Problem Parsers}


\subsection{Linear Temporal Logic Inference Engine}
Our LTL inference engine is based on the $progress$ function from 
\cite{nau2004apt}. Applying an operator during the HTN planning process changes 
the state of the world, which necessitates a call to the $progress$ function. 
The $progress$ function simultaneously evaluates the current set of control 
rules in the new state, and derives the set of control rules that will be 
applicable to the next state encountered. It relies on the same atom entailment 
and variable binding procedures used during the HTN planning process.

In addition to the $progress$ function, we were required to implement a 
function to be called whenever a solution plan is found, that verifies that the 
final world state is a valid one. The semantics of the $\lozenge p$ 
(``eventually $p$'') expression require that $p$ become true at some point, at 
which point $\lozenge p$ disappears from the control rule. As such, an 
$\lozenge p$ expression present in the control rules progressed through the 
final state can potentially invalidate the solution plan.

\subsection{Limitations} 
The LTL constraint computations in TL-SHOP are not compiled in the same way 
that the HTN problems were in JSHOP2. In fact, we feel that the progressions of 
LTL formulas computed by the LTL inference engine do not lend themselves to 
compilation in any obvious way. As such, the LTL inference engine in TL-SHOP 
takes a considerable performance hit because it needs to allocate and 
deallocate memory during the planning process, which JSHOP2 managed to avoid.

The LTL operators $\forall$ (``forall'') and $\exists$ (``exists'') take as 
their first argument not an unbound variable, but instead a logical atom that 
contains a single unbound variable. Ideally, this atom has a small number of 
ground instances, and so the combinatorial explosion generally produced by 
these statements is controlled.

TL-SHOP does not support function symbols in LTL expressions, as we lacked 
enough time to properly implement such functionality.

\section{Examples}


\section{Conclusion}
TL-SHOP does what we set out to do---combine Hierarchal Task Network planning 
with the control rule planning paradigm in such a way as to increase domain and 
problem specification flexibility. It currently lags far behind JSHOP2 in 
performance, but this could be remedied by future work which discovers a way to 
extend JSHOP2's domain and problem compilation process to include progressions 
of LTL formulas.

\bibliographystyle{plain}
\bibliography{final_paper}

\end{document}